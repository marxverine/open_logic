% Part: computability
% Chapter: computability-theory
% Section: computable-sets

\documentclass[../../../include/open-logic-section]{subfiles}

\begin{document}

\olfileid{cmp}{thy}{cps}
\olsection{Computable Sets}

We can extend the notion of computability from computable functions to
computable sets:

\begin{defn}
  Let $S$ be a set of natural numbers. Then $S$ is \emph{computable}
  iff its characteristic function is. In other words, $S$ is
  computable iff the function
\[
\Char{S}(x) =
\begin{cases}
1 & \text{if $x \in S$} \\
0 & \text{otherwise}
\end{cases}
\]
is computable. Similarly, a relation $\Atom{R}{x_0, \dots, x_{k-1}}$ is
computable if and only if its characteristic function is.
\end{defn}

\begin{explain}
Computable sets are also called \emph{decidable}.

Notice that we now have a number of notions of computability: for
partial functions, for functions, and for sets. Do not get them
confused!{} \iftag{TMs}{The Turing machine computing a partial function
  returns the output of the function, for input values at which the
  function is defined; the Turing machine computing a set returns
  either 1 or 0, after deciding whether or not the input value is in
  the set or not.}{}
\end{explain}

\begin{prob}
  Consider an arbitrary finite set of naturals $A$. Prove that $A$ is computable.
\end{prob}

\begin{prob}
  Consider an arbitrary finite relation $B$ on the naturals. Prove that $B$ is computable.
\end{prob}

\begin{prob}
  Consider the relation $<$ on the naturals. Notice that it is an infinite set, so it is not obviously computable; but it is computable! Prove that $<$ is computable.
\end{prob}

\begin{prob}
  Let $k$ be some positive integer. Prove that the union of $k$ computable sets (of natural numbers) is a computable set. Also prove that the union of $k$ computable relations (on the naturals) is a computable relation.
\end{prob}

\begin{prob}
  There are problems where one makes a yes-or-no decision. We intuitively call these problems "decision problems". In computability theory, one can represent many decision problems as sets. In this problem, we examine one such instance. We consider a simplified version of the blackjack card game. \newline
  Let $d$ be some positive integer. In our simplified gane, let us say that there are $d$ decks of cards, where every deck contains 52 cards (there are no joker cards). Each card is assigned a natural value from 1 to 13. For example, an ace is going to have a value of 1. The numbered cards are going to have the value printed on them (for example, a 2 card is going to have a value of 2). The king, queen, and jack cards each have a value of 10. All $52d$ cards are shuffled together in a pile. \newline
  In our simplified game, we have two moves: "hit" and "stand". When you call "hit", a random card is drawn from the pile and is added to your hand. You are then allowed to call "hit" as many times as you want, until cards run out from the pile. Whenever you call "stand" (or when you are forced to call "stand" because you cannot call "hit" anymore), you look at your hand as it is; you count the total value of cards in your hand. If the value is at most 21, then we say that the hand is a "winning hand." (Recall that we are playing a simplified version of blackjack, so this will suffice.) \newline
  Formally, we can represent a non-empty "hand" (of cards that you are holding) as an element of the following set: $HANDS = \bigcup_{x \in \mathbb{N}, x>0} ( \mathbb{N}^x )$. Let $WINNING$ be the set of all winning hands for our game. Observe that $WINNING$ is a subset of $HANDS$. Now, prove that $WINNING$ is a computable set. (Note: by proving this, we will show that the decision problem of "whether a hand is a winning hand" is a problem that computers can solve.)
\end{prob}

\end{document}
